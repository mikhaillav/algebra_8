\section{Основные определения и способы задания}

\begin{definition}[Множество]
	Конечная или бесконечная совокупность попарно различных объектов.
\end{definition}

\begin{remark}
	Множества принято обозначать заглавными латинскими буквами.
\end{remark}

\begin{definition}[$a \in A$]
	Объект $a$ является элементов множества $A$.
\end{definition}

\subsection{Способы задания множеств}

\subsubsection{1. Перечисление}

\begin{example}
	$A = \{1; 2; 3\}$
\end{example}

\begin{example}
	$A = \{\text{Кошечки}; \text{Собачки}; \text{Ученики 30ки}\}$
\end{example}

\subsubsection{2. С помощью предиката}

\begin{example}
	$A = \{x \mid x \; \vdots \; 5\}$ ($A$ принадлежат все $x$ которые делают из предиката $x \; \vdots \; 5$ истинное высказывание )
\end{example}

\begin{example}
	$A = \{(x; y) \mid x \; \vdots \; 5 \land y \; \vdots \; 4\}$ ($A$ принадлежат пары $x$ и $y$ которые делают из предиката истинное высказывание )
\end{example}

\subsubsection{3. Числовые промежутки}

$(a; b) = \{x \mid x > a \land x < b \}$ - интервал \\
$(a; b] = \{x \mid x > a \land x \le b \}$ - полуинтервал \\
$[a; b] = \{x \mid x \ge a \land x \le b \}$ - отрезок \\

$
\left.
\begin{array}{l}
	\left( a; +\infty \right) = \left\{x \mid x > a \right\} \\
	\left[ a; +\infty \right) = \left\{x \mid x \ge a \right\} \\
	\left( -\infty; b \right) = \left\{x \mid x < b \right\} \\
	\left( -\infty; b \right] = \left\{x \mid x \le b \right\}
\end{array}
\right\} \quad \text{- лучи }
$

\subsubsection{4. Спец. символы}

$\mathbb{N} = \left\{ 1; 2; 3; 4 \dots \right\}$ - множество натуральных чисел \\
$\mathbb{Z} = \left\{ 0; \pm 1; \pm 2; \pm 3; \pm 4 \dots \right\}$ - множество целых чисел \\
$\mathbb{Q}$ - множество рациональных чисел \\
$\mathbb{R}$ - множество вещественных чисел \\ 
$\varnothing$ - пустое множество

\begin{definition}[Равных множеств]
	Два множества называются равными между собой если они состоят из одних и тех же элементов.
\end{definition}

\begin{definition}[Подмножество]
	Говорят, что множество $A$ является подмножеством множества $B$, если каждый элемент множества $A$ является так же и элементом множества $B$.
\end{definition}

\begin{remark}
	Пустое множество является подмножеством любого множества (в том числе и пустого)
\end{remark}

