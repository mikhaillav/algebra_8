\chapter{Квадратные уравнения}

\begin{definition}[Квадратное уравнение]
	Квадратным уравнением называется уравнение вида $ax^2 + bx + c = 0$, где $a \neq 0$. 
\end{definition}

\section{Способы решения}

\subsection{Удачные случаи}

Если $b = 0$ и $c = 0$ то все довольно просто:

\begin{align*}
	& ax^2 = 0 \iff \\
	& \left[\begin{array}{l}
		a = 0 \text{ - так не бывает}\\
		x^2 = 0
	\end{array}\right. \iff \\
	& x^2 = 0 \iff \\
	& x = 0
\end{align*}

Если $b = 0$, а у $a$ и $c$ разные знаки (иначе у нас получится сумма двух положительных чисел которая не может быть равна нулю) то решаем так:

\begin{align*}
	& ax^2 - c = 0 \iff \\
	& (\sqrt{a}x - \sqrt{c})(\sqrt{a}x + \sqrt{c}) \iff \\
	& \left[\begin{array}{l}
		\sqrt{a}x - \sqrt{c} = 0 \\
		\sqrt{a}x + \sqrt{c} = 0
	\end{array}\right. \iff \\
	& \sqrt{a}x = \pm \sqrt{c} \iff \\
	& x = \pm \sqrt{\frac{c}{a}}
\end{align*}

Если $c = 0$ то делаем так:

\begin{align*}
	& ax^2 + bx = 0 \iff \\
	& x(ax + b) = 0 \iff \\
	& \left[\begin{array}{l}
		x = 0 \\
		ax + b = 0
	\end{array}\right. \iff \\
	& \left[\begin{array}{l}
		x = 0 \\
		x = - \frac{b}{a}
	\end{array}\right.  
\end{align*}

\newpage

\subsection{Остальные случаи}

В остальных случаях стоит воспользоваться следующими теоремами.

\begin{theorem}[О решении квадратных уравнений]
	Пусть имеется квадратное уравнение и \\ $D = b^2 - 4ac$. \\
	Тогда утверждается, что если:
	\begin{enumerate}
		\item $D < 0$:
			Уравнение не имеет решений.
		\item $D = 0$:
			$x = - \frac{b}{2a}$
		\item $D > 0$:
			$x = - \frac{b \pm \sqrt{D}}{2a}$
	\end{enumerate}
\end{theorem}

\begin{proof}
	\begin{align*}
		& ax^2 + bx + c = 0 \iff \\
		& x^2 + \frac{bx}{a} + \frac{c}{a} = 0 \iff \\
		& x^2 + 2 \frac{b}{2a} x + \frac{b^2}{4a^2} - \frac{b^2}{4a^2} + \frac{c}{a} = 0 \iff \\
		& (x + \frac{b}{2a})^2 - (\frac{b^2}{4a^2} - \frac{c}{a}) = 0 \iff \\
		& (x + \frac{b}{2a})^2 - \frac{b^2 - 4ac}{4a^2} = 0 \iff \\
		& (x + \frac{b}{2a})^2 - \frac{D}{4a^2} = 0 \\
	\end{align*}
	
	\begin{enumerate}
		\item $D < 0$:
		
			Сумма неотрицательного $(x + \frac{b}{2a})^2$ и положительного -$\frac{D}{4a^2}$ числа не может быть равна нулю.
			
		\item $D = 0$:
		
			\begin{align*}
				& (x + \frac{b}{2a})^2 = 0 \iff \\
				& x + \frac{b}{2a} = 0 \iff \\
				& x = - \frac{b}{2a}
			\end{align*}
			
		\item $D > 0$:
		
			\begin{align*}
				& (x + \frac{b}{2a} - \frac{\sqrt{D}}{2a})(x + \frac{b}{2a} + \frac{\sqrt{D}}{2a}) = 0 \iff \\
				& x + \frac{b}{2a} \pm \frac{\sqrt{D}}{2a} = 0 \iff \\
				& x = -\frac{b}{2a} \pm \frac{\sqrt{D}}{2a} \iff \\
				& x = - \frac{b \pm \sqrt{D}}{2a}
			\end{align*}
	\end{enumerate}
\end{proof}

\pagebreak

 todo: написать формулировку прямой и обратной на мат языке

\begin{theorem}[Виета]
	Числа $x_1$, $x_2$ тогда и только тогда являются корнями квадратного уравнения, для них выполняются следующие соотношения:
	$$
	\left\{\begin{array}{l}
		x_1 + x_2 = - \frac{b}{a} \\
		x_1 * x_2 = \frac{c}{a}
	\end{array}\right.
	$$
\end{theorem}

\begin{proof}
	
\end{proof}