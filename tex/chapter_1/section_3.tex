\section{Импликация и эквиваленция}

\begin{definition}[Импликация]
	Импликацией высказываний $A$ и $B$ называется высказывание, обозначаемое $A \rightarrow B$, (произносится как <<если А то Б>>, <<А достаточно для Б>>), истинность которого определяется таблицей:
	\begin{center}
		\begin{tabular}{ |c|c|c| } 
			\hline
			$A$ & $B$ & $A \rightarrow B$ \\
			\hline 
			1 & 1 & 1 \\ 
			1 & 0 & 0 \\
			0 & 1 & 1 \\ 
			0 & 0 & 1 \\  
			\hline
		\end{tabular}
	\end{center}
\end{definition}

\begin{definition}[Эквиваленция]
	Эквиваленцией высказываний $A$ и $B$ называется высказывание, обозначаемое $A \leftrightarrow B$, (произносится как <<А тогда и только тогда когда Б>>, <<А необходимо и достаточно для Б>>), истинность которого определяется таблицей:
	\begin{center}
		\begin{tabular}{ |c|c|c| } 
			\hline
			$A$ & $B$ & $A \leftrightarrow B$ \\
			\hline 
			1 & 1 & 1 \\ 
			1 & 0 & 0 \\
			0 & 1 & 0 \\ 
			0 & 0 & 1 \\  
			\hline
		\end{tabular}
	\end{center}
\end{definition}

\subsection{Свойства импликации}

\begin{theorem}[Выражение через другие логические операции] \label{thm:1.3.1}
	$A \rightarrow B = \overline{A} \lor B$
\end{theorem}

\begin{proof}
	\hfill \break \break
	\begin{center}
		\begin{tabular}{ |c|c|c|c| } 
			\hline
			$A$ & $B$ & $A \rightarrow B$ & $\neg A \lor B$ \\
			\hline 
			1 & 1 & 1 & 1 \\ 
			1 & 0 & 0 & 0 \\
			0 & 1 & 1 & 1 \\ 
			0 & 0 & 1 & 1 \\  
			\hline
		\end{tabular}
	\end{center}
\end{proof}

\begin{theorem}[Отрицание импликации]
	$\overline{A \rightarrow B} = A \land \overline{B}$
\end{theorem}

\begin{proof}
	Воспользуемся теоремой \ref{thm:1.3.1} выше, получаем, что
	\begin{align*}
		\overline{A \rightarrow B} = \overline{A} \lor B 
	\end{align*}
	Затем, используя теорему \ref{thm:1.2.4} (об отрицании конъюнкции), преобразуем в
	\begin{align*}
		\overline{A} \lor B = A \land \overline{B}
	\end{align*} 
	
\end{proof}

\newpage

\begin{theorem}[Обоснование доказательства от противного]
	$A \rightarrow B = \overline{B} \rightarrow \overline{A}$
\end{theorem}

\begin{proof}
	\hfill \break \break
	\begin{center}
		\begin{tabular}{ |c|c|c|c| } 
			\hline
			$A$ & $B$ & $A \rightarrow B$ & $\neg B \rightarrow \neg A$ \\
			\hline 
			1 & 1 & 1 & 1 \\ 
			1 & 0 & 0 & 0 \\
			0 & 1 & 1 & 1 \\ 
			0 & 0 & 1 & 1 \\  
			\hline
		\end{tabular}
	\end{center}
\end{proof}

\begin{theorem}[Транзитивность импликации]
	\hfill \break
	\begin{align*}
		\left\{\begin{array}{l}
			A \rightarrow B \\
			B \rightarrow C
		\end{array}\right. \rightarrow (A \rightarrow C)
	\end{align*}
\end{theorem}

\begin{proof}
	Для доказательства данного свойства нам важно показать, что импликация посередине (между системой и второй импликацией) будет выполняться всегда, то есть ее столбик будет заполнен единичками.
	
	\begin{center}
		\begin{tabular}{ |c|c|c|c|c|c| } 
			\hline
			$A$ & $B$ & $C$ & $\overbrace{(A \rightarrow B) \land (B \rightarrow C)}^{D}$ & $\overbrace{(A \rightarrow C)}^{E}$ & $D \rightarrow E$ \\
			\hline 
			1 & 1 & 1 & 1 & 1 & 1 \\ 
			1 & 1 & 0 & 0 & 0 & 1 \\ 
			1 & 0 & 1 & 0 & 1 & 1 \\ 
			1 & 0 & 0 & 0 & 0 & 1 \\
			0 & 1 & 1 & 1 & 1 & 1 \\ 
			0 & 1 & 0 & 0 & 1 & 1 \\ 
			0 & 0 & 1 & 1 & 1 & 1 \\ 
			0 & 0 & 0 & 1 & 1 & 1 \\ 
			\hline
		\end{tabular}
	\end{center}
\end{proof}