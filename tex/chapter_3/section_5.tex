\section{Виды теорем}

\begin{definition}
    Пусть исходная (прямая) теорема это $A(x) \Rightarrow B(x)$, тогда: \\
    $B(x) \Rightarrow A(x)$ - обратная теорема; \\
    $\overline{A(x)} \Rightarrow \overline{B(x)}$ - противоположная теорема; \\
    $\overline{B(x)} \Rightarrow \overline{A(x)}$ - обратная к противоположной теорема.
\end{definition}

\begin{example}
    \hfill

    Прямая: $\alpha$ и $\beta$ - вертикальные $\Rightarrow$ они равны \\
    Обратная: $\alpha$ и $\beta$ - равны $\Rightarrow$ они вертикальные \\
    Противоположная: $\alpha$, $\beta$ - не вертикальные $\Rightarrow$ они не равны \\ 
    Обратная к противоположной: $\alpha$, $\beta$ - не равны $\Rightarrow$ они не вертикальные
\end{example}

\begin{theorem}[О видах теорем]
    \hfill

    \begin{enumerate}
        \item Прямая и обратная к противоположной теореме теоремы всегда верны или не верны одновременно.
        \item Обратная и противоположная теоремы всегда верны или не верны одновременно
    \end{enumerate}
\end{theorem}

\begin{proof}
    Согласно свойству импликации \ref{thm:1.3.3}:

    \begin{enumerate}
        \item $A(x) \Rightarrow B(x) = \forall x \; A(x) \rightarrow B(x) = \forall x \; \overline{B(x)} \rightarrow \overline{A(x)} = \overline{B(x)} \Rightarrow \overline{A(x)}$
        \item $B(x) \Rightarrow A(x) = \forall x \; B(x) \rightarrow A(x) = \forall x \; \overline{A(x)} \rightarrow \overline{B(x)} = \overline{A(x)} \Rightarrow \overline{B(x)}$
    \end{enumerate}
\end{proof}