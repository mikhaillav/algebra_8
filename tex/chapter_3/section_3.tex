\section{Кванторы и двухместные предикаты}

\begin{remark}[Как читаются выражения с двумя кванторами]
    \hfill

    \begin{enumerate}
        \item $\forall x \; \forall y$ - для каждого $x$ каждый $y$
        \item $\forall x \; \exists y$ - для каждого $x$ найдется $y$
        \item $\exists x \; \forall y$ - найдется $x$, для которого каждый $y$
        \item $\exists x \; \exists y$ - найдутся такие $x$ и $y$
    \end{enumerate}
\end{remark}

\subsection{Как ими пользоваться?}

$P(x; y)$ - предикат, зависящий от $x$ и $y$ \\
$\forall x \; P(x; y)$ - предикат, зависящий от $y$ \\
$\exists x \; P(x; y)$ - предикат, зависящий от $y$ \\
$\forall y \; P(x; y)$ - предикат, зависящий от $x$ \\
$\exists y \; P(x; y)$ - предикат, зависящий от $x$ \\

$\forall x \; \forall y \; P(x; y)$ - высказывание, зависящие от порядка кванторов \\
$\forall x \; \exists y \; P(x; y)$ - высказывание, зависящие от порядка кванторов \\

\begin{theorem}[Об отрицании выскзаываний с несколькими кванторами]
    \hfill \break

    $\overline{\forall x \; \exists y \; \forall z \; P(x; y; z)} \leftrightarrow \exists x \; \forall y \; \exists z \; \overline{P(x; y; z)}$
\end{theorem}

\begin{proof}
    Поочередно раскрываем каждый из кванторов согласно теореме \ref{thm:3.2.7}.
    \begin{align*}
        &\overline{\forall x \; \exists y \; \forall z \; P(x; y; z)} \leftrightarrow \\
        &\forall x \; \overline{\exists y \; \forall z \; P(x; y; z)} \leftrightarrow \\
        &\exists x \; \forall y \; \overline{\forall z \; P(x; y; z)} \leftrightarrow \\
        &\forall x \; \forall y \; \exists z \;  \overline{P(x; y; z)} \leftrightarrow \\
    \end{align*}
\end{proof}