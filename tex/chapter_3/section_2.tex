\section{Кванторы}

\begin{definition}[Квантор общности]
    Квантором общности называется символ <<$\forall $>> (<<для всех>>, <<для каждого>>, <<для любых>>) который при употреблении перед предикатом $P(x)$
    превращает его в высказывание $\forall x \; P(x)$ истинное тогда и только тогда, когда $D(P) = T(P)$
\end{definition}

\begin{example}
    $\forall x \; x = x$ -- истина
\end{example}

\begin{example}
    $\forall x \; x^2 \ge 0$ -- истина
\end{example}

\begin{example}
    $\forall a \in \mathbb{Z} \; a \; \vdots \; 2 \rightarrow a \; \vdots \; 4 $ -- ложь
\end{example}

\begin{definition}[Квантор существования]
    Квантором существования называется символ <<$\exists $>> (<<найдется>>, <<существует>>,) который при употреблении перед предикатом $P(x)$
    превращает его в высказывание $\exists x \; P(x)$ истинное тогда и только тогда, когда $T(P) \ne \varnothing $
\end{definition}

\begin{example}
    $\exists x \; x = 1$ -- истина
\end{example}

\begin{example}
    $\exists x \; x^2 \ge 0$ -- истина
\end{example}

\begin{example}
    $\exists a \in \mathbb{Z} \; a \; \vdots \; 2 \rightarrow a \; \vdots \; 4 $ -- истина
\end{example}

\newpage

\begin{theorem}[Об отрицании высказываний, содержащих кванторы] \label{thm:3.2.7}
    \hfill \break \break
    1) $\overline{\forall x \; P(x)} \leftrightarrow \exists x \; \overline{P(x)}$ \\ \\
    2) $\overline{\exists x \; P(x)} \leftrightarrow \forall x \; \overline{P(x)}$
\end{theorem}

\begin{proof}
    (1) \\
    \begin{align*}
        &\overline{\forall x \; P(x)} \leftrightarrow \\
        &\overline{D(P) = T(P)} \leftrightarrow \\
        &D(P) \ne T(P) \leftrightarrow \\
        &\exists x \in D(P)\;  x \notin T(P) \leftrightarrow \tag{*} \\
        &\exists x \in D(\overline{P})\;  x \notin T(P) \leftrightarrow \\
        &\exists x \in D(\overline{P})\;  x \in T(\overline{P}) \leftrightarrow \tag{**} \\
        &T(\overline{P}) \ne \varnothing \leftrightarrow \\
        &\exists x \; \overline{P(x)}
    \end{align*}

    \textsuperscript{*} $D(P) = D(\overline{P})$

    \textsuperscript{**} если найдется такой $x$, что $x \in T(\overline{P})$, то это значит, что $T(\overline{P})$ не пустое
\end{proof}

\begin{proof}
    (2) \\
    \begin{align*}
        &\overline{\exists x \; P(x)} \leftrightarrow \\
        &\overline{T(P) \ne \varnothing} \leftrightarrow \\
        &T(P) = \varnothing \leftrightarrow \tag{*}\\
        &T(\overline{P}) = D(P) \leftrightarrow \tag{**} \\
        &T(\overline{P}) = D(\overline{P}) \leftrightarrow \\
        &\forall x \; \overline{P(x)}\\
    \end{align*}

    \textsuperscript{*} если ничего не может превратить предикат в истинное высказывание, значит все значения переменной которые вообще превратят предикат 
    в высказывание превратят его в ложное высказывание

    
    \textsuperscript{**} $D(P) = D(\overline{P})$
\end{proof}

\newpage