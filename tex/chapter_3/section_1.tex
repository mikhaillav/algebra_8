\begin{remark}
    Предикат <<$P$>>, зависящий от переменной $x$ обозначается как $P(x)$.
\end{remark}

\section{Два множества у предиката}

\begin{definition}[Область определения предиката]
    ООП -- множество тех значений переменной при подстановке которых предикат становится высказыванием. Для предиката <<$P$>> ООП обозначается как $D(P)$
\end{definition}

\begin{example}
    $P(x) = \text{<<} x \; \vdots \; 5 \text{>>}$ ; $D(P) = \mathbb{Z}$
\end{example}

\begin{example}
    $P(x) = \text{<<} \frac{1}{x} = 0 \text{>>}$ ; $D(P) = \mathbb{R} \backslash \{0\}$
\end{example}

\begin{definition}[Область истинности предиката]
    ОИП -- множество тех значений переменной при подстановке которых предикат становится истинным высказыванием. Для предиката <<$P$>> ОИП обозначается как $T(P)$
\end{definition}

\begin{example}
    $P(x) = \text{<<} x \ge 5 \land x \le 10 \text{>>}$ ; $D(P) = \mathbb{R}$ ; $T(P) = \left[ 5; 10 \right]$
\end{example}
